\chapter{Conclusion}
\label{chap:conclusion}
En conclusion, ce rapport offre une exploration complète du problème du plus court chemin et du problème du voyageur de commerce, en mettant l'accent sur l'implémentation pratique et l'évaluation des performances de diverses approches algorithmiques. L'étude démontre l'efficacité de l'algorithme A* pour le SPP, en montrant sa capacité à trouver des chemins optimaux dans un graphe ; avec une complexité computationnelle réduite lorsqu'il est amélioré avec des fonctions heuristiques et des structures de données telles que les files de priorité et les ensembles. Pour le TSP, le rapport souligne l'impraticabilité des méthodes de force brute pour les graphes de grande taille en raison de leur complexité temporelle exponentielle, tout en montrant l'efficacité de l'outil d'optimisation CPLEX pour résoudre des instances plus importantes de manière efficiente. L'analyse comparative souligne l'importance de choisir le bon algorithme en fonction de la taille du problème et des contraintes. Ce travail renforce non seulement la compréhension théorique des algorithmes de graphe, mais fournit également des perspectives intéressantes sur leurs applications pratiques et leurs limitations.
