\chapter{Plus court chemin}
\label{ch:shortest_path}

\section{Modélisation}
\label{sec:shortest_path_model}

On considère un graphe non orienté $G=<S,A>$ où $S$ est l'ensemble des sommets et $A$ l'ensemble des arêtes. Chaque arête $a_{ij}$ est associée à un coût $c_{ij}$, qui vaudra 1 dans le cas où deux sommets sont reliés horizontalement ou verticalement, et $\sqrt{2}$ dans le cas où ils sont reliés en diagonale. On cherche à déterminer le plus court chemin entre un sommet de départ $s$ et un sommet d'arrivée $t$.

\subsection*{Variables}

\begin{itemize}
    \item $x_{ij}$ : vaut 1 si l'arête $a_{ij}$ est empruntée, 0 sinon
\end{itemize}

\subsection*{Fonction objectif}

On cherche à minimiser la somme des coûts des arêtes empruntées :

\begin{equation}
    \min \sum_{(i,j) \in A} c_{ij} \cdot x_{ij}
\end{equation}

\subsection*{Contraintes}

\begin{itemize}
    \item Le sommet de départ $s$ est toujours relié à un sommet :
    \begin{equation}
        \sum_{j \in S} x_{sj} = 1
    \end{equation}
    \item De même, le sommet d'arrivée $t$ est toujours relié à un sommet :
    \begin{equation}
        \sum_{i \in S} x_{it} = 1
    \end{equation}
    \item Le sommet de départ $s$ n'a pas d'arête entrante :
    \begin{equation}
        \sum_{i \in S} x_{is} = 0
    \end{equation}
    \item De même, le sommet d'arrivée $t$ n'a pas d'arête sortante :
    \begin{equation}
        \sum_{j \in S} x_{tj} = 0
    \end{equation}
    \item Chaque sommet a le même nombre d'arêtes entrantes et sortantes (sauf $s$ et $t$):
    \begin{equation}
        \sum_{j \in S} x_{ij} = \sum_{j \in S} x_{ji} \quad \forall i \in S \setminus \{s,t\}
    \end{equation}
    \item Notre graphe n'étant pas orienté, nous devons empêcher les sous-cycles, c'est-à-dire le cas où on trouve une arête $a_{ij}$ et une arête $a_{ji}$ dans le chemin :
    \begin{equation}
        \sum_{(i,j) \in A} x_{ij} + \sum_{(j,i) \in A} x_{ji} \leq 1 \quad \forall i,j \in S \setminus \{s,t\}
    \end{equation}
\end{itemize}

Nous avons implémenté et résolu ce problème en Python, en utilisant la librairie \texttt{docplex.mp.model} de CPLEX. Le code complet est disponible en annexe \ref{app:shortest_path_code}.

\section{A Star}
\label{sec:shortest_path_astar}

\section{Génération de graphes aléatoires}
\label{sec:shortest_path_random_graph}

Afin de pouvoir comparer les deux implémentations, nous avons créé une fonction générant des graphes aléatoires. Cette fonction prend en paramètre le nombre de sommets $n$ et la probabilité $p$ qu'un sommet soit un obstacle.

\begin{figure}[H]
    \centering
    \begin{includegraphics}[width=1\textwidth]{resources/gen_path_graph.png}
    \end{includegraphics}
    \caption{Fonction générant un graphe aléatoire}
    \label{fig:gen_path_graph}
\end{figure}

\section{Résultats et comparaison des deux implémentations}
\label{sec:shortest_path_comparison}

Les deux méthodes étant fondamentalement différentes, nous pouvons observer de légères différences sur les résultats obtenus. Par exemple, sur le graphe \texttt{reseau\_20\_20\_1}, nous avons une différence :

\begin{figure}[H]
    \centering
    \begin{includegraphics}[width=.6\textwidth]{resources/20_20_cplex.png}
    \end{includegraphics}
    \caption{Graphique de la solution trouvée par CPLEX}
    \label{fig:cplex_2020}
\end{figure}

\begin{figure}[H]
    \centering
    \begin{includegraphics}[width=.6\textwidth]{resources/20_20_astar.png}
    \end{includegraphics}
    \caption{Graphique de la solution trouvée par A Star}
    \label{fig:astar_2020}
\end{figure}