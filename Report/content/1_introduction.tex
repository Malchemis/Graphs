\chapter{Introduction}
\label{chap:introduction}
Dans le domaine de l'informatique et de la cybersécurité, l'étude de la théorie des graphes et de l'optimisation joue un rôle important, en particulier pour résoudre des problèmes complexes de routage et de recherche de chemin. Ce rapport se penche sur deux problèmes classiques dans ce domaine : le problème du plus court chemin et le problème du voyageur de commerce. Ces problèmes sont non seulement fondamentaux pour la recherche théorique mais ont également des applications pratiques significatives dans des domaines tels que le routage des réseaux, la logistique et la robotique. Ce travail pratique vise à explorer des solutions algorithmiques efficaces pour ces problèmes, en utilisant des approches classiques et heuristiques. Ce rapport détaille la structure du projet, les méthodologies employées, l'implémentation en Python et une analyse comparative des résultats obtenus à partir de différentes stratégies algorithmiques.